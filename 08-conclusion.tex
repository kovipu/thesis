\chapter{Conclusion} \label{conclusion}

The laws regarding storing individuals' private data in effect in Finland require additional technical knowledge and work from developers dealing with such data.
Healthcare organizations are forced to comply with the requirements and possibly implement additional security measures to their digital services.
The GDPR also adds extra measures to using a third party hosting provider.
Users of any public cloud provider have to take extra caution no third party can gain access to the sensitive data.
The individual must also be fully informed on where and how their data is stored and processed.

The GDPR and the Finnish law do not go into specifics regarding the technical implementation.
However, encryption and pseudonymization of data is encouraged to keep the data safe from a malicious actor.
These requirements have to be considered when implementing a digital service that stores any private data.

\section{The results of the research}

The proof-of-concept project built to mimic a real world healthcare service shows clear results on how encryption affects a cloud-based application's implementation and performance. Implementing encryption requires additional knowledge on cryptography from the development team.
However, the developer of a web application should almost never create and deploy their own cryptographic algorithm.\cite{dont-roll-crypto}
Libraries and tools exist for implementing encryption which should be preferred over custom cryptography implementations.
Nevertheless, implementing encryption will increase the complexity of the application.

In the case of the POC project created for this thesis, a lot of the complexity is handled by existing solutions such as Google Cloud Platform’s Cloud Key Management and Node.js’ built-in crypto-library.
However, implementing encryption increased the codebase size of both the application and the infrastructure by quite a bit.
This increased complexity makes the codebase harder to understand and therefore increases the cost to maintain it.

The added computation required by the encryption also affected the performance of the application.
Load testing shows a clear increase in response times when encryption is enabled.
Similarly, the throughput of the application decreased with encryption.
However, this difference between the two tests is not as large as it might be assumed.
The application is most likely more limited by its I/O.
The performance hit might still affect the costs of hosting the application in a public cloud.

\section{Options for improvement}

The POC project was built to mimic a real world healthcare service’s use case and implementation.
However, it is not feature-complete enough to be deployed in to the real world.
All the tested aspects of the application have room for improvement.

The security of the application could be further improved in a few different ways.
The data is only being kept safe from a third party gaining access to the database, as Google still has access to all the keys required to decrypt it.
This could be improved by storing any of the keys required to decrypt the data outside of the Google Cloud.

In the POC project, the key encryption key is stored inside program memory during runtime.
This responsibility could be moved to the Cloud Key Management by using its API to encrypt and decrypt the data encryption keys.
This would make it harder for a malicious actor to gain access to both, the KEK and the DEK.
In a real world use case, the KEK would also have to be rotated regularly, which was not considered in the POC implementation.
The secrets could also be handled separately from the Terraform configuration, making sure they are not committed to version control in either Terraform’s configuration or state files.

The added complexity of implementing encryption could also be handled better.
The application could use another layer of abstraction handling the encryption.
This abstraction could be implemented between the business logic and the database layer, making reasoning about the application logic on a high level easier without having to consider the underlying encryption.
The infrastructure configuration could separate encryption-specific configuration to its own modules.

The performance of the application can also be improved.
The application was found to handle 2000 simultaneous requests quite well while running in a single process.
The server application is stateless, which means it can natively scale into multiple processes in a fully managed service such as GCP’s Cloud Run.
The database’s performance has space for performance improvement as well -- perhaps by adding a fast in-memory cache.

The \texttt{GET /practitioners/} endpoint failing could be improved with a better design.
In a real world scenario, loading all the rows from an entire database table is not a good idea.
Instead, the client and the server should be designed to use pagination.
This reduces the bottleneck in loading data, as the data is not loaded all at once.
