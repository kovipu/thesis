\chapter{Patient data in a public cloud} \label{data security}

This chapter covers laws effective in Finland on storing individual's health data.
The jurisdiction is analyzed from the point of view of a healthcare provider storing patient data in a public cloud.
The laws are analyzed more by their technical requirements rather than their juridical ones.

\section{Patient data security laws in Finland}

The Finnish law includes multiple legislations to follow when storing patient data.
The main laws on this topic are for example Data Protection Act\footnote{Tietosuojalaki} 1050/2018, Act on the Status and Rights of Patients\footnote{Laki potilaan asemasta ja oikeuksista} 785/1992 and Act on the Electronic Processing of Client Data in Healthcare and Social Welfare\footnote{Laki sosiaali- ja terveydenhuollon asiakasti- etojen sähköisestä käsittelystä} 784/2021.
The main legislation to follow on data protection in Finland, which the Finnish laws complement and expand on, is the General Data Protection Regulation of the European Union.
\cite{valvira}

The General Data Protection Regulation (EU) (GDPR) is a regulation on data protection and security in the European Union (EU).
GDPR applies to all enterprises storing the data of an individual living inside the European Economic Area (EEA), regardless whether the processing itself takes place in the Union.
GDPR aims to protect individuals' data and enhance their control over it, while also making the regulation simpler to follow for companies.
\cite{gdpr}

\section{Technical requirements of the laws in Finland}

The GDPR requires the data controller --- an organization that collects data from EU residents --- to provide the individual with information and options.
The GDPR also gives some specifics on how to handle the data securely.
Many of these mandates may be fulfilled using technology.
\cite{gdpr}

The GDPR mandates that the data controller is responsible for notifying the user and collecting consent before collecting any personal data.
The individual must explicitly give consent for example by ticking a box. Silence, inactivity or pre-ticked boxes do not count as consent.
The consent must also be stored in a way that the data controller can prove that they have collected it.
\cite{gdpr}

After collecting consent, the individual has the right to know whether their data is being processed.
The consent may also be given or withdrawn at any time at will.
It should be just as easy to withdraw consent as it is to give it.
At the point the data is collected, the user must be informed of the time period the data will be stored.
If the data is no longer needed for the purpose it was collected for, it should be deleted.
The data controller must also provide a copy of the individual's data when requested.
The exported data must be in ''a structured and commonly used and machine-readable'' format.
\cite{gdpr}

The GDPR also mandates a right to be forgotten for the individual.
This means the individual can request the removal of all data concerning him or her.
The individual may also request the data to be transferred to another data controller.
The controller must oblige to these requests without any undue delay.
\cite{gdpr}

Collecting parental consent is also required by the GDPR.
Child age is defined by the specific member state, in the GDPR regulations the default being 16 years. \cite{gdpr}
In the case of Finland the specific age is 13 years \cite{tietosuojalaki}. 
In Finland, a child can use advisory, support or preemptive services without their parent's consent \cite{tietosuojavaltuusto}.

Handling a Finnish resident's social security number\footnote{Sosiaaliturvatunnus} (SSN) is regulated by The Data Protection Act.
Permission for processing the individual's SSN is granted for identification of the individual for legal purposes, such as healthcare.
The SSN must not be printed redundantly on any documents.
\cite{tietosuojalaki}

The GDPR specifies some guidelines on data security.
While not too technical, these guidelines give organizations good pointers on how to handle data.
Data security standards should be judged on a per-case basis, but some of the things the GDPR suggests are \cite{gdpr}:
\begin{itemize}
    \item the pseudonymization and encryption of personal data;
    \item the ability to ensure the ongoing confidentiality, integrity, availability and resilience of processing systems and services;
    \item the ability to restore the availability and access to personal data in a timely manner in the event of a physical or technical incident;
    \item a process for regularly testing, assessing and evaluating the effectiveness of technical and organizational measures for ensuring the security of the processing.
\end{itemize}

The data security measures can be confirmed appropriate by following an approved code of conduct or by an approved certification mechanism.
Each member state should provide a public authority to supervise GDPR compliance.
This supervisory authority has a broad and sweeping power over data controllers' activities.
\cite{gdpr}
In Finland, this supervisory authority is The Office of the Data Protection Ombudsman\footnote{Tietosuojavaltuutetun toimisto}. \cite{tietosuojavaltuutettu}

\section{Requirements for storing patient data}

The GDPR is very far from actual technical implementation, and only suggests the developer to consider the pseudonymization and encryption of personal data regarding the implementation.
The GDPR suggests considering the level of security required for data storage based on the nature of the data as well as the risks related to any incidents.
An individual's health data is considered "a special category of personal data" so the developer should take extreme caution when dealing with said data, as it is sensitive by nature.
\cite{gdpr}

Complying with the GDPR regulations will take a fair amount of technology, even if just for the bookkeeping.
Not only must data controllers make individuals’ personal data transparent and editable, they must make records of the individuals’ wishes available to supervisory authorities or else face sanction.
Any of these failures is punishable by a significant fine. \cite{gdpr}
\begin{itemize}
    \item Failure by a Controller or Processor, or their representative, to provide information on request to a supervisory authority required for the performance of their tasks.
    \item Failure by a Controller or Processor to provide access to all personal data or information necessary for performance of supervisory authority tasks.
    \item Failure to allow access to premises, including any data processing equipment.
    \item Failure to comply with an order to comply with an individual's requests.
    \item Failure to comply with an order to bring processing in to compliance in a specified manner and in a specified period.
    \item Failure to comply with an order to communicate a personal data breach to individuals.
    \item Failure to comply with a prohibition on processing.
    \item Failure to comply with an order to rectify, restrict, or erase data and to notify 3rd parties of such actions.
    \item Failure of a Certifying Body to comply with an order to cease issuing certifications.
    \item Failure to comply with an order to cease transfers of data to 3rd countries or an international organization.
\end{itemize}

Validating the privacy practices of an organization GDPR-compliant is not a well-defined task.
GDPR says to follow a code of conduct, but such codes are hard to come by.
Getting a certification from a supervisory authority is another way to validate an organization's GDPR compliance.
\cite{gdpr}

\section{GDPR compliance in a public cloud}

Public cloud is characterized by providing computation, storage \& networking services outside of one's current organization.
It also offers near infinite scalability and on-demand deployments at an inexpensive price.
Companies have adopted the public cloud at a fast speed in recent years – a trend that is probably not going to be reversed.
For example, Amazon Web Services (AWS) provides cloud computing infrastructure to over 1 million organizations in 190 countries.
\cite{sevensins}

As modern computing systems focus on performance, cost-efficiency, reliability, and scalability, not many organizations give enough thought to security and privacy.
The rise of the GDPR forces organizations to face the issues on privacy and security.
Usage of a public cloud complicates these problems even more, as multiple organizations can share computing and networking resources.
\cite{sevensins}

While a multitenant cloud has a lot of benefits for the organization using it, sharing a virtualized pool of computing and networking services is also a possible cause for security issues.
When multiple organizations share the same resource, it is a lot more likely for personal data to leak or for an unauthorized party to gain access to the data.
Public cloud providers' lack the tools required to monitor and audit these types of security issues, and leave a lot of the problems for the developer to deal with.
\cite{cloudauditing}

Right to be forgotten is not a simple task to adhere to in the real world.
Not only must all manually collected backups be cleared of the individual's data, the cloud provider may also have taken copies of the data for performance, security and scalability reasons.
Google reports that their cloud services can take up to 180 days to get the data fully deleted.
\cite{sevensins}

Cloud services are quite opaque in terms of where the data is being processed.
The individual might not know where their data is sent when using such services.
This is especially a problem for multi-layer cloud services, where parts of the infrastructure might be handled by an entirely separate cloud services, for purposes such as handling payments and collecting analytics.
The GDPR requires organizations using cloud to solve this issue by giving precise information on where the individual's data is being sent when asking for consent.
Knowing where the data is handled might not be clear to the developer either.
\cite{sevensins}

Public cloud services do also provide quite a few tools for hardening a service for better privacy and security.
Cloud providers allow separating resources into virtual private networks with configurable firewalls between the network and other resources inside the cloud or the internet.
Public cloud also has tooling for implementing for example public key cryptography and Google Cloud encrypts all data stored on their cloud storage at rest by default. 
\cite{googlecloud}
